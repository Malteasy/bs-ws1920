\documentclass[11pt, oneside]{article}   	% use "amsart" instead of "article" for AMSLaTeX format
\usepackage{geometry}                		% See geometry.pdf to learn the layout options. There are lots.
\geometry{letterpaper}                   		% ... or a4paper or a5paper or ... 
%\geometry{landscape}                		% Activate for rotated page geometry
%\usepackage[parfill]{parskip}    		% Activate to begin paragraphs with an empty line rather than an indent
\usepackage{graphicx}				% Use pdf, png, jpg, or eps§ with pdflatex; use eps in DVI mode
								% TeX will automatically convert eps --> pdf in pdflatex		
\usepackage{amssymb}
\usepackage[latin1]{inputenc}r
%SetFonts

%SetFonts


\title{�bung 2}
\author{Malte R�der, Lukas Sparenberg, ...}
%\date{}							% Activate to display a given date or no date

\begin{document}
\maketitle
\section{Aufgabe 1}
a) 
\\ \\ 
Speicheradresse des Stacks: beginnend 0x7, endend mit der Zahl 4
\\
Speicheradresse des Heaps: 0x5, endend mit der Zahl 4
\\ \\
Speicheradresse des Stacks ist h�her als die des Heaps. 
Die erste und die letzte Stelle des Stacks und des Heaps der Ausgaben in unterschiedlichen Terminals sind jeweils gleich.
\\ \\
F�r Stack und der Heap werden durch die Ausf�hrung mittels zweier Terminals auch zwei Bereiche f�r den Ablauf der separat gestarteten Programme alloziiert, so dass n�tige Ressourcen wie programm counter etc. genutzt werden k�nnen.
\\ \\
\\ \\
b)
\\ \\
Beobachtung:
\\ \\
Speicheradressen sind entweder beide gr��er oder kleiner (im Vergleich der Terminals) 
Synchrones Schreiben und Lesen von und in Register, durch 2 gleichzeitig gestartete Prozesse mit 60 sek. sleep in denen beide Prozesse im busy waiting Modus sind. 
Mechanismus: Signalisierung.

c)
\\ \\
Speicheradresse bleiben pro run Aufruf gleich.
Durch die Nutzung des Debuggers, bleibt der Prozess im selben Bereich, w�hrend die normale Durchf�hrung immer wieder neue Speicherbereiche anlegt.
\\ \\
\newpage
d) 
\\ \\
Man w�rde erwarten, dass durch den Aufruf im rechten Terminal der Wert der Speicheradresse des linken Terminalaufrufs ver�ndert bzw. �berschrieben wird. Da der Prozess jedoch in einem anderen Block (PCB) wartet (sleep) kann auf diese Speicheradresse nicht zugegriffen werden.

\end{document}  